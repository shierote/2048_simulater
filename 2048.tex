\documentclass[11pt,a4paper]{jsarticle}
\usepackage{amsmath,amssymb}
\usepackage{newtxtext,newtxmath}
\usepackage[dvipdfmx]{graphicx}
\usepackage{listings}
\lstset{%
  language={Ruby},
  % backgroundcolor={\color[gray]{.95}},%
    tabsize=2, % tab space width
    showstringspaces=false, % don't mark spaces in strings
    basicstyle={\ttfamily},%
    %identifierstyle={\small},%
    commentstyle={\itshape},%
    keywordstyle={\bfseries},%
    %ndkeywordstyle={\small},%
    stringstyle={\ttfamily},
    %frame={tb},
    breaklines=true,
    columns=[l]{fullflexible},%
    % numbers=left,%
    % numberstyle={\small},%
    xrightmargin=0zw,%
    %xleftmargin=3zw,%
    stepnumber=1,
    numbersep=1zw,%
    lineskip=-0.5ex%
}

\title{2048攻略プログラム}

\author{江口大志(J5-170079)、陶山 大輝(J5-170188)}

%
\begin{document}
\maketitle
%

\section{2048について}
\noindent
%2048がどのようなゲームであるかについて書く
2048はイタリア人のガブリエレ・チルリによって2014年に公開されたパズルゲームです。\\
このゲームでは、4×4のマスに数字が書かれたタイルがあり、スライドさせると全てのタイルがマスの端まで移動して新たなタイル(2または4)が出現し、同じ数字のタイルがぶつかると数字が足し合わされて一つのタイルに変化します。\\
タイルの数字を増やしていき2048のタイルを作ることができればゲームクリアですが、それ以後も続けることができます。また、タイルをスライドさせることができなくなるとゲームオーバーになります。\\


\section{プログラムについて}
\subsection{プログラムの目標}
\noindent
%最大どのくらいまで点は伸びるのか
%2048でできるだけ点を増やすにはどうするか、増やす方法はどのように実装するか
僕たちは、このゲームにおいて、できるだけ大きな数字のタイルを作り出すプログラムを作ろうと考えました。\\
まず、このプログラムにおいて生まれうる最大の数字のタイルは131072($2^{17}$)です。(下記の参考サイトを参照しました。)\\
この数字にできるだけ近い数を作り出そうと試行錯誤して出来上がったのがこのプログラムです。\\
ちなみに、人間の最高得点は、ネットで調べた限りだと32768($2^{17}$)のようです。\\

\newpage
\subsection{プログラムの講義との関連性}
\noindent
このプログラムは金子先生のゲームプログラミングの授業と関連があります。\\
授業で扱ったゲーム木探索やミニマックス探索を用いたプログラムを実装しました。\\
強化学習による実装は間に合いませんでしたが、強化学習を用いればより高い点数を出すことができそうです。\\



\section{プログラムのコードについて}

\subsection{ソースコード}
%この中でsusubsectionを作ったほうがいい気がする
\lstset{numbers=left}
\begin{lstlisting}

\end{lstlisting}
\subsection{ソースコードの説明}

\noindent
%それぞれのコードで何をしているかの説明
\subsection{プログラムの検証}
\noindent
%このプログラムは点を増やすのにどのような工夫が用いられているのか、果たしてそれは有効なのか
%この方法よりも上のプログラムはありうるか、それともないのか、ならばそれはなぜか

\section{参考サイト等}
\noindent
https://qiita.com/amoO\_O/items/743715e918c87d4c4930
→2048における最高点の計算

\end{document}
