\documentclass[11pt,a4paper]{jsarticle}
\usepackage{amsmath,amssymb}
\usepackage{newtxtext,newtxmath}
\usepackage[dvipdfmx]{graphicx}
\usepackage{listings}
\lstset{%
  language={Ruby},
  % backgroundcolor={\color[gray]{.95}},%
    tabsize=2, % tab space width
    showstringspaces=false, % don't mark spaces in strings
    basicstyle={\ttfamily},%
    %identifierstyle={\small},%
    commentstyle={\itshape},%
    keywordstyle={\bfseries},%
    %ndkeywordstyle={\small},%
    stringstyle={\ttfamily},
    %frame={tb},
    breaklines=true,
    columns=[l]{fullflexible},%
    % numbers=left,%
    % numberstyle={\small},%
    xrightmargin=0zw,%
    %xleftmargin=3zw,%
    stepnumber=1,
    numbersep=1zw,%
    lineskip=-0.5ex%
}

\title{2048攻略プログラム}

\author{江口大志(J5-170079)、陶山 大輝(J5-170188)}

%
\begin{document}
\maketitle
%

\section{2048について}
\subsection{2048の説明}
\noindent
%2048がどのようなゲームであるかについて書く
\subsection{2048の考察}
\noindent
%最大どのくらいまで点は伸びるのか
%2048でできるだけ点を増やすにはどうするか、増やす方法はどのように実装するか


\section{プログラムについて}
\subsection{ソースコード}
\lstset{numbers=left}
\begin{lstlisting}

\end{lstlisting}
\subsection{ソースコードの説明}
\noindent
%それぞれのコードで何をしているかの説明
\subsection{プログラムの検証}
\noindent
%このプログラムは点を増やすのにどのような工夫が用いられているのか、果たしてそれは有効なのか
%この方法よりも上のプログラムはありうるか、それともないのか、ならばそれはなぜか

\section{参考サイト等}
\noindent

\end{document}
